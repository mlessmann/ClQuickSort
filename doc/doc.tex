\documentclass{llncs}

\usepackage[utf8]{inputenc}
\usepackage[english]{babel}

\usepackage{algorithm}
\usepackage{algorithmicx}
\usepackage{algpseudocode}
\usepackage{booktabs}
\newcommand{\doctype}{Parallel Quick Sort in OpenCL}

\usepackage{hyperref}
\hypersetup{
  breaklinks=true,
  colorlinks=true,
  citecolor=blue,
  linkcolor=blue,
  urlcolor=blue,
  bookmarksnumbered,
  bookmarksopen,
  pdftitle={\doctype},
  pdfauthor={Patrick Bisenius, Manuel Leßmann},
  pdfsubject={},
  pdfkeywords={},
}

% Final Submit TODO: remove following line which makes margins smaller
\hypersetup{pdfpagescrop={92 62 523 748}}

\usepackage{breakurl}

\title{\doctype}
\author{Patrick Bisenius, Manuel Leßmann}
\institute{Karlsruhe Institute of Technology, Karlsruhe, Germany}

\begin{document}

\maketitle

\begin{abstract}
We have implemented a parallel version of the Quick Sort algorithm using OpenCL to be executed on GPUs. Experiments show, that a significant speedup compared to the sequential CPU version is only possible for very large inputs. A combination with a GPU algorithm that performs better for smaller inputs is advisable to maximize speedup.
\end{abstract}

\section{Algorithm}
Since the sequential Quick Sort algorithm is inherently recursive in nature, we used a similar approach for our GPU version. In every recursive call a pivot is selected and the GPU swaps elements such that smaller elements than the pivot are to its left in the array and larger elements are to its right. This is done by computing two prefix sums - one for the lower elements, one for the upper elements. An element adds 1 to the respective prefix sum, if it is lower/higher than the pivot, otherwise 0. As a result the number in the prefix sum array at the position of an element is its target index in the output array. Afterwards we only have to fill the gap between lower and upper numbers with pivots.

Pseudocode for this is shown in Algorithm~\ref{alg1}. It is written in SIMD-style such that the function is called on a new thread for every item in the input. The parameter $i$ is the index of the current item. The parameters $offset$ and $count$ describe the bounds of the current recursion; they start at $0$ and input-length respectively.

\begin{algorithm}[h]
\begin{algorithmic}
\Function{QuickSort}{$input, output, offset, count, i$}
\State $pivot \gets$ Choose pivot
\State $lps \gets$ Left prefix sum array
\State $rps \gets$ Right prefix sum array
\If{$input[offset + i] < pivot$}
  \State $lps[i] \gets 1$
\ElsIf{$input[offset + i] > pivot$}
  \State $rps[i] \gets 1$
\EndIf\\
\State \Call{PrefixSum}{$lps$}
\State \Call{PrefixSum}{$rps$}
\State $countLeft \gets lps[last]$
\State $countRight \gets rps[last]$
\State $countPivots \gets count - countLeft - countRight$\\
\If{$input[offset + i] < pivot$}
  \State $output[offset + lps[i]] \gets input[offset + i]$
\ElsIf{$input[offset + i] > pivot$}
  \State $output[offset + countLeft + countPivots + rps[i]] \gets input[offset + i]$
\EndIf
\If{$i < numberPivots$}
  \State $output[offset + countLeft +i] \gets pivot$
\EndIf
\State $input \gets output$
\State \Call{QuickSort}{$input, output, offset, countLeft, i$}
\State \Call{QuickSort}{$input, output, offset + countLeft + countPivot, countRight, i$}
\EndFunction
\end{algorithmic}
\label{alg1}
\caption{SIMD-pseudocode for the parallel quick sort algorithm on GPUs.}
\end{algorithm}

\section{Implementation Details}
We ended up implementing inclusive prefix sums to compute the new indices, because we needed to know the total amount of elements left and right of the pivot. This information is missing in an exclusive prefix sum. Therefore the new index for a number is now stored one element lower in the prefix sum result, and we needed an exception for the very first element, which reads from index 0.

We also made extensive use of local memory to speed up read and write operations. For instance, in the first step of each recursion we need to compute the number the elements to the left and right of the pivot. This is done by the kernel "CountElements". Every work group counts for themselves and writes its results into a global array. In the next step we use an inclusive prefix sum on these arrays to compute the starting indices for the values of each work group as well as the total number of elements to the left, to the right and equal to the pivot. The last kernel "DistributeElements" then computes for every element its new index using the proper value of the prefix sum and adding a local offset value that gets incremented atomically. As a last step the indices between the left and right groups need to be filled with the pivot value.

In principle our algorithm is the same for 32-bit and 64-bit integers as well as floating point numbers. Copying the kernel and replacing the relevant data types or, more elegantly, C++-Style templates would very easily make the algorithm type-independent. However, there is no proper way to write templates for OpenCL kernels. Therefore, our algorithm only supports 32-bit integers. On top of that, many GPUs only support 64-bit integers by software emulation. This support has not yet made it into the standard OpenCL API and is only accessible through vendor-specific extensions.

\section{Experimental Results}

We benchmarked our algorithm on a machine with an i5-4670, 8GB of RAM and an AMD Radeon R290. We used \texttt{std::uniform\_int\_distribution} to generate a random array of numbers.

Table~\ref{tbl:input_size} compares the runtime of \texttt{std::sort} and our implementation for different input sizes. Our implementation has a large amount of overhead, especially for small input sizes. However, it scales better than \texttt{std::sort} for larger input sizes because more and more elements in each step are equal to the pivot and are discarded in further recursion steps.

Table~\ref{tbl:unique_numbers} shows the impact of different amounts of unique numbers for a constant input size of $2^{20}$. With more unique numbers, less and less elements will be equal to the pivot element in each step. This increases the size of the remaining blocks and therefore the number of necessary recursion steps for our algorithm. The last column shows the runtime of our algorithm if the recursion is stopped if the block size is smaller than 1024.

We did not achieve a speedup with respect to \texttt{std::sort} for any input size. As shown in Table~\ref{tbl:unique_numbers}, a large percentage of the total runtime is spent in recursion steps with block sizes smaller than 1024 elements. For these small blocks, the overhead for the kernel invocations and copy operations on the OpenCL buffers becomes too large. Since OpenCL kernels are invoked from a command queue and only one kernel is executed at a time, the number of active GPU threads becomes smaller and smaller for each recursion step. This problem could be solved by implementing a second sorting algorithm for small block sizes that uses a single workgroup to sort a small block and sorts multiple small blocks at a time, such as bitonic sort. 

\begin{table}
\parbox{.45\linewidth}{
\centering

\begin{tabular}{@{}lll@{}}
\toprule
n         & CPU  & GPU   \\ \midrule
$2^{20}$    & 73   & 10151 \\
$2^{21}$    & 140  & 10183 \\
$2^{22}$   & 284  & 11050 \\
$2^{23}$    & 548  & 11134 \\
$2^{24}$  & 1091 & 11093 \\
$2^{25}$   & 2133 & 11809 \\
$2^{26}$   & 4263 & 12909 \\
$2^{27}$ & 8620 & 13764 \\ \bottomrule
\end{tabular}
\caption{Runtime in milliseconds of the QuickSort algorithm with $2^{14}$ different numbers.}
\label{tbl:input_size}
}
\hfill
\parbox{.45\linewidth}{
\centering
\begin{tabular}{@{}lrrr@{}}
\toprule
\# numbers       & CPU & GPU   & rec. aborted    \\ \midrule
$2^{14}$   & 74  & 10934 & 393 \\
$2^{18}$  & 82  & 37087 & 393 \\
$2^{20}$ & 84  & 76149 & 373 \\ \bottomrule
\end{tabular}
\caption{Runtime in milliseconds for $n=2^{20}$ and different amounts of unique numbers. The last column shows the runtime if the recursion is aborted when the block size is smaller than $1024$ to show the impact of the last recursion steps.}
\label{tbl:unique_numbers}
}
\end{table}


%\bibliographystyle{splncs03}
%\bibliography{/home/axtman/promotion/latex/library}

\end{document}
